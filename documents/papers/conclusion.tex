\section{Conclusion}

% your personal appreciation if any

It is clear that utilizing the full potential of the processor is crucial for
high performance. Being able to utilize that power it should be made easier for
programmers to write parallel programs. Most programmers are good at writing
sequential programs but writing good parallel programs can be very difficult.
Even though the numbers of cores keep increasing, the speedup is also very much
dependent on the ratio of the program that can be parallelized
\cite{hennessy2007computer}. The four papers summarized in section~\ref{papers}
describe different programming models that can help achieving this, but
\cite{CaoPerformanceAnalysis} showed that not all programming models scale that
well to many-core architectures.

Many of the parallel programming models discussed in the papers describe how
you can rewrite existing algorithms to be distributed across the cores of a
processor. Usually this is done by sprinkling some extra language constructs or
library calls, for example instead a \texttt{for} statement a
\texttt{cilk\_for} statement. Maybe instead of creating the algorithm
sequentially first and then trying to parallelize it should change to begin
with parallelism in mind from the start with a thoroughly concurrent
programming language.
