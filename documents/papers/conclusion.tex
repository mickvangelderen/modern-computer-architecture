\section{Conclusion}

% your personal appreciation if any

It is clear that utilizing the full potential of the processor is crucial for
high performance. Being able to utilize that power it should be made easier for
programmers to write parallel programs. Most programmers are good at writing
sequential programs but writing good parallel programs can be very difficult.
Even though the numbers of cores keep increasing, the speedup is also very much
dependent on the ratio of the program that can be parallelized
\cite{hennessy2007computer}. The four papers summarized in section~\ref{papers}
describe different programming models that can help achieving this, but
\cite{CaoPerformanceAnalysis} showed that not all programming models scale that
well to many-core architectures.

Many of the parallel programming models discussed in the papers describe how
you can rewrite existing algorithms to be distributed across the cores of a
processor. Usually this is done by sprinkling some extra language constructs or
library calls, for example instead a \texttt{for} statement a
\texttt{cilk\_for} statement. Maybe the answer does not lie in finding ways to parallelize sequential algorithms but in educating programmers to think sequentially. An other approach towards the same goal is finding ways of programming that allow parallelization which are more intuitive to programmers. Event based programs for example are for a lot of programmers much easier to understand. If we manage to find efficient algorithms for effectively finding parallelizable events, better parallelization can be achieved while it is essentially coming from the programming level. 

The ideal model is one that is easy to use and learn while having the effective computational power scale linearly with the amount of cores. 
The PPM model seems to be closest to that from the models we have discussed and it is definitely a good starting point for further developments. 
However, the industry seems to be heavily focussed on bringing the technique from GPU's to the CPU and maybe we need to take an entirely different approach. 