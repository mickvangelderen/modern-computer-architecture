\section{Conclusion}

All in all the lab was definitely fun. It felt really rewarding to hack
something together, inject some bytecode and see it working.

We've identified a kernel function and changed the x264 program so the kernel
is executed on the \rvex. The kernel we used is the \satd function. Using our
abstracted \texttt{rvex*} functions you can write the instruction memory, write
and read the data memory and start the \rvex. Unfortunately we didn't see a
speedup, instead we even saw a decrease in performance. Presumably this is
because of the communication overhead. The \satd function is called many times.
This can improved by reducing the number of reads and writes to the \rvex. This
means selecting a bigger part of the application and send it over to the
\rvex in its entirety.

Using our \texttt{rvex*} functions, we were able to create different test
programs which helped us testing the communication and get to know the
platform, getting the endianness right and fixing multiple executions of the
\rvex.

Regarding the results, we have some ideas that we would have tried out given
enough time. We are interested in if the \rvex performance is dependent on the
type of program that is run. Is it more efficient for highly repeating code or
does it perform just as well for dynamic code with little repetition compared
to the normal CPUs?
