\section{Profiling}

To know what we are going to optimize, we need to profile the application
first. Just taking a random function isn't a good idea. Fortunately we can
compile the x264 program with the gprof profiling flags enabled. Compiling the
program and running it on the Virtual Machine with the
$\sim$/Videos/inputs/eledream\_640x360\_8.y4m as input video we get the
following profiling results:

\begin{small}
\begin{lstlisting}
gprof x264 | head -n 10
Flat profile:

Each sample counts as 0.01 seconds.
  %   cumulative   self              self     total
 time   seconds   seconds    calls  ms/call  ms/call  name
 13.70      0.10     0.10  1599044     0.00     0.00  x264_pixel_satd_8x4
 13.70      0.20     0.10   570708     0.00     0.00  get_ref
  6.85      0.25     0.05    38770     0.00     0.00  x264_pixel_sad_x4_16x16
  5.48      0.29     0.04   460484     0.00     0.00  quant_4x4
  4.11      0.32     0.03   123076     0.00     0.00  sa8d_8x8
\end{lstlisting}\end{small}

We see that the function \satd{} is called 1.6 million times during the video
conversion. These calls together take up about $14\%$ of the total time. An
equal amount of time is spent in \satd{}.

The video eledream\_640x360\_128.y4m makes the application spend about $18\%$
of the runtime in the pixel processing function and $16\%$ in \getref{}.

The bigger our input video the more time we will spend processing and the more
effect an optimization will have if it targets part of the processing code.

Judging from the profiling information there are two potential places where
optimization will be effective: \satd{} and \getref{}.

When we looked at the source code of x264 we thought that the nature of \satd{}
was more suitable for optimization because it had some loops and arithmetic in
it. The \getref{} function was a lot more irregular and also harder to
understand. On the flip side, the overhead caused by communication between the
MicroBlaze processor and the \rvex{} would be smaller for \getref{} because it
is called three times less than \satd{}.

In the end we chose to try and put the computation of \satd{} on the
co-processor and let \getref{} for what it was.  The function \satd{} was
easier to understand and making it work was more important to us than choosing
the best optimization area right away.

