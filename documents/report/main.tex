\documentclass{article}
\usepackage[utf8]{inputenc}

\title{Modern Computer Architecture\\Lab Report}
\author{%
    Mick van Gelderen\\4091566
    \and
    Arian Stolwijk\\4001079
}

\date{November 2013}

\usepackage{natbib}
\usepackage{graphicx}
\usepackage{listings}
\usepackage{xcolor}

\definecolor{code-bg}{gray}{0.96}
\definecolor{code-comment}{HTML}{056121}

\lstdefinestyle{C}{%
    language=C,
    backgroundcolor=\color{code-bg},
    basicstyle=\ttfamily\small,
    keywordstyle=\color{blue}\ttfamily,
    stringstyle=\color{red}\ttfamily,
    commentstyle=\color{code-comment}\ttfamily,
    showstringspaces=false,
    tabsize=2
}

\lstdefinestyle{bash}{%
    language=bash,
    backgroundcolor=\color{code-bg},
    basicstyle=\ttfamily\small,
    keywordstyle=\color{blue}\ttfamily,
    stringstyle=\color{red}\ttfamily,
    commentstyle=\color{code-comment}\ttfamily,
    showstringspaces=false,
    tabsize=2
}

\newcommand{\rvex}{\ensuremath{\rho}-VEX}
\newcommand{\satd}{\texttt{x264\_pixel\_satd\_8x4}}
\newcommand{\getref}{\texttt{get\_ref}}

\begin{document}

\maketitle


\begin{abstract}
    For the practical assignment of the Modern Computer Architecture course
    we've changed the x264 program. We've profiled x264 and extracted the
    \satd{} as kernel function. To get a performance improvement we've moved
    the execution of the kernel function to the \rvex{}, a co-processor of the
    MicroBlaze. A requirement for this is to load the program into the
    instruction memory and write and read the data to and from the data memory.
    We've found that our implementation did not generate a performance
    increase, but rather a decrease. We think this is due to the communication
    overhead.
\end{abstract}




\section{Introduction}

The assignment for this report was basically, find a kernel function from an
application and run it on the \rvex{}. To do this there are a few things we
need to know first:

\paragraph{Platform} The platform the program will run on is the ERA platform.
It consists out of the host processor, the MicroBlaze, accelerated with a
VLIW co-processor, the \rvex{}. For the co-processor computationally intensive
kernels can be extracted to achieve a performance increase.

\paragraph{Kernel function} The extracted piece of code is called the kernel.
The kernel needs to be compiled for the \rvex{} so that we can inject the
result, the bytecode, in to the co-processor. The rest of the code runs as
usual on a regular processor.

\paragraph{x264} The x264 program is the application we will try to improve
using the \rvex{}. x264 is a software library for encoding video streams
in the H.264 compression format.

The goal for this report is to extract the correct kernel, using profiling and
compile this for the \rvex{}. To execute the kernel on the \rvex{} there should
be communication between the processor and the co-processor.

\section{Profiling}

To know what we are going to optimize, we need to profile the application
first. Just taking a random function isn't a good idea. Fortunately we can
compile the x264 program with the gprof profiling flags enabled. Compiling the
program and running it on the Virtual Machine with the
$\sim$/Videos/inputs/eledream\_640x360\_8.y4m as input video we get the
following profiling results:

\begin{small}
\begin{lstlisting}
gprof x264 | head -n 10
Flat profile:

Each sample counts as 0.01 seconds.
  %   cumulative   self              self     total
 time   seconds   seconds    calls  ms/call  ms/call  name
 13.70      0.10     0.10  1599044     0.00     0.00  x264_pixel_satd_8x4
 13.70      0.20     0.10   570708     0.00     0.00  get_ref
  6.85      0.25     0.05    38770     0.00     0.00  x264_pixel_sad_x4_16x16
  5.48      0.29     0.04   460484     0.00     0.00  quant_4x4
  4.11      0.32     0.03   123076     0.00     0.00  sa8d_8x8
\end{lstlisting}\end{small}

We see that the function \satd{} is called 1.6 million times during the video
conversion. These calls together take up about $14\%$ of the total time. An
equal amount of time is spent in \satd{}.

The video eledream\_640x360\_128.y4m makes the application spend about $18\%$
of the runtime in the pixel processing function and $16\%$ in \getref{}.

The bigger our input video the more time we will spend processing and the more
effect an optimization will have if it targets part of the processing code.

Judging from the profiling information there are two potential places where
optimization will be effective: \satd{} and \getref{}.

When we looked at the source code of x264 we thought that the nature of \satd{}
was more suitable for optimization because it had some loops and arithmetic in
it. The \getref{} function was a lot more irregular and also harder to
understand. On the flip side, the overhead caused by communication between the
MicroBlaze processor and the \rvex{} would be smaller for \getref{} because it
is called three times less than \satd{}.

In the end we chose to try and put the computation of \satd{} on the
co-processor and let \getref{} for what it was.  The function \satd{} was
easier to understand and making it work was more important to us than choosing
the best optimization area right away.


\section{Data Layout}

The \satd{} function has the following signature:

\begin{verbatim}
int x264_pixel_satd_8x4(
    pixel *pix1, int i_pix1,
    pixel *pix2, int i_pix2);
\end{verbatim}

This is a function that takes two arrays of pixels, something that has to do
with the indices of the pixels in the array, and returns a single integer.

From the implementation we can see that the data used by the array is always $4
\times 8 = 32$ pixels per array. This means that the data layout should have at
least two times 32 times the size of one pixel. The size of a pixel is defined
as \texttt{uint\_8}. If we choose \texttt{i\_pix1} and \texttt{i\_pix2} to
always be 8, we don't have to send those values, which saves sending data.
Setting those parameters to a fixed value has also the advantage that the array
indexes looked up by the function are always predictably 0 to 32

Also for the return value we should reserve 4 bytes, which is the size of an
integer. When we have calculated the result, we could write this back to the
first position, but we write the result after the data of the two pixel arrays.
The advantage for doing this is because it's slightly easier and clearer for future maintainers. 


\section{Implementation}

We have access to several (but not enough during the lab) FPGAs which are
configured as MicroBlaze processors and run Linux.

The \rvex{} is a configured as a co-processor that can be controlled using a
number of memory mapped files. The is a file for writing the instruction
memory, one for reading and writing the data memory, one for reading the status
and one for writing control commands.

We abstracted this away to a small interface with the following functionality:

\begin{description}
    \item[rvexInit] \hfill \\
        Attempts to open the IMEM, DMEM and SMEM files. You can also specify
        the bytecode which will be written to the instruction memory.
    \item[rvexDispose] \hfill \\
        Closes all files opened by rvexInit.
    \item[rvexWrite] \hfill \\
        Allows you to write to the data memory.
    \item[rvexRead] \hfill \\
        Allows you to read from the data memory.
    \item[rvexSeek] \hfill \\
        Allows you to jump to a given position in the data memory.
    \item[rvexGo] \hfill \\
        Writes the clear and start commands to the control memory and blocks
        until the status reports that the operation was successful.
\end{description}

\subsection{Endianness}

Since we had to compile for the MicroBlaze using the flag
\texttt{-DWORD-BIGENDIAN} we figured that the MicroBlaze would be a big endian
machine. You can always test it by writing a multibyte value like
\texttt{0xDEADBEEF} to memory and read the individual bytes. If you read
\texttt{0xDE 0xAD 0xBE 0xEF} you will know that you have a big endian machine.
If you get that sequence but in reverse you know its a little endian machine.

We write the data as big endian to the data memory. For types bigger than one
byte we change the endianness on the \rvex{} side using a macro. Also before
writing data to the memory from the \rvex{} the endianness should be changed.
For the \satd{} kernel the size of the pixels is one byte, only the result is
an integer, so four byes, which should be reversed.

\subsection{Loading Instruction memory}

Loading the instruction memory with separate commands before executing the
program on the MicroBlaze is pretty inconvenient. That is why we write the
bytedata to the instruction memory in the application. The compiled kernel
function is actually converted to a piece of C code and saved in a file which can then directly be written
to the instruction memory. This solves the inconvenience of multiple files for
the same program.

\section{Results of using the \rvex{}}

We modified the x264 application to log its processing time computed with
\texttt{clock\_gettime}. This required linking the rt library.

Then we tried to find a FPGA that was not being used by anyone else and we ran
several versions of the x264 application.  All of them included the code from
Listing~\ref{lst:timing} which allowed us to get an idea of the total runtime.

\begin{lstlisting}[language=C,style=C,caption=Capturing runtime with the monotonic clock,label=lst:timing]
struct timespec tss, tse, tsd; // start, end and diff
clock_gettime(CLOCK_MONOTONIC, &tss);

if( !ret )
    ret = encode( &param, &opt );

clock_gettime(CLOCK_MONOTONIC, &tse);
if (tse.tv_nsec > tss.tv_nsec) {
    tsd.tv_sec = tse.tv_sec - tss.tv_sec;
    tsd.tv_nsec = tse.tv_nsec - tss.tv_nsec;
} else {
    tsd.tv_sec = tse.tv_sec - tss.tv_sec - 1;
    tsd.tv_nsec = tse.tv_nsec - tss.tv_nsec + 1000000000;
}
printf("Took %lu.%09lu sec\n", tsd.tv_sec, tsd.tv_nsec);
\end{lstlisting}

The versions that we created were the following

\begin{description}
    \item[vanilla] \hfill \\
        The original x264 implementation.
    \item[rvex simple] \hfill \\
        Simple implementation where memory mapped files are opened once for stability during execution and only one \rvex{} is supported at a time.
    \item[rvex inline] \hfill \\
        Based on simple, uses an object to store the file descriptors so allows multiple \rvex{} processors to be used and the time critical functions are placed in a header file with the inline annotation.
\end{description}

We ran the script from Listing~\ref{lst:testscript} to get an idea of the runtimes of the different versions.
Note that we already did manual test runs to get an idea of how long the runs would take.
We noticed that running the same test multiple times produced very similar run times.
We did not find it necessary to do multiple runs for the purposes of this lab because the error would be very small in comparison with the run times for different versions of the application:

\begin{verbatim}
Took 75.280923348 sec
Took 75.350194581 sec
Took 75.221714038 sec
Took 75.374499116 sec
\end{verbatim}

\begin{lstlisting}[style=bash,caption=Test script,label=lst:testscript]
echo 'eledream 64x36 3 frames with timing' >> group11.log
./x264-timing-sb2 eledream_64x36_3.y4m -o out.mkv | grep Took >> group11.log

echo 'eledream 640x360 8 frames with timing' >> group11.log
./x264-timing-sb2 eledream_640x360_8.y4m -o out.mkv | grep Took >> group11.log

echo 'eledream 64x36 3 frames with rvex interface simple' >> group11.log
./x264-rvex-sb2 eledream_64x36_3.y4m -o out.mkv | grep Took >> group11.log

echo 'eledream 64x36 3 frames with rvex interface struct and inline' >> group11.log
./x264-rvex-struct-inline-sb2 eledream_64x36_3.y4m -o out.mkv | grep Took >> group11.log
\end{lstlisting}

\begin{lstlisting}[style=bash,caption=Test output,label=lst:testoutput]
/ # cat group11.log
eledream 64x36 3 frames with timing
Took 75.699494266 sec
eledream 640x360 8 frames with timing
Took 317.733194410 sec
eledream 64x36 3 frames with rvex interface simple
Took 442.169817935 sec
eledream 64x36 3 frames with rvex interface struct and inline
Took 440.886382661 sec
\end{lstlisting}

Table~\ref{tab:exe-times} summarizes the results found by manual testing and the listed script.

\begin{table}[!h]
    \centering
    \caption{Execution times}
    \label{tab:exe-times}
    \begin{tabular}{|c|c|c|}
       \hline
                   & 64x36 3f & 640x360 8f \\ \hline
       vanilla     & 75.7s    & 317s       \\
       rvex simple & 442s     & $>$20m     \\
       rvex inline & 441s     & untested   \\
       \hline
    \end{tabular}
\end{table}

Most importantly, we see that our optimization actually made the application
run several times slower.  Also, we can conclude that the gcc compiler does a
terrific job at inlining functions when \texttt{-O3} is enabled, even if you do
not ask it to.  We couldnt wrap our heads around doing so much communication
related tasks in a function that was basicly a relatively small fixed number of
basic operations (*, +, -, shifts, comparisons) and achieving a faster result
by that.  We assumed that the \rvex was extremely tightly integrated into the
MicroBlaze which allowed very good communication speeds but the test results
show that this is not the case.

The speedup that we obtained can be calculated by looking at the time spent in
the kernel function. This should be about 14\% for the input video
eledream\_640x360\_8.y4m judging from the profiling results obtained on the
VM\@. We cannot say exactly how much time is spent there because we cannot
profile on the MicroBlaze.

Because the \rvex simple version took longer than 20 minutes which caused the
SSH session to hang we cannot use the exact run time. Let us see what the
speedup would be if the run time was 20 minutes exactly.

The vanilla version spends 14\% of the 75 seconds that it runs in \satd{}. So
86\% is spent in the rest of the functions. Since we have not changed anything
beside the kernel we can assume that the time spent in that part will be equal
for vanilla and for rvex simple. This means that $\frac{1200 - 0.86\cdot75}{75 -
0.86\cdot75} = 108$ is the relative speed of the vanilla function in comparison
with the rvex version.  It actually runs 108 times faster.  In other words, the
rvex version runs 108 times slower which amounts to a speedup of -10700\%
obtained by introducing rvex.

Theoretically, the perfect optimalization for the selected kernel would be to reduce the computation time to 0.
This means that the 14\% becomes 0\% and so the maximum attainable speedup is $\frac{100}{100 - 14} - 100 = 16\%$

\section{Additional Assigment}
Except for the joy of achieving an actual speedup, we did not think that extracting another function would teach us a whole lot more about the process.
It would in our eyes teach us a lot about the x264 application but not without severe frustration and confusion as to what is going on.

So we set a challenge for ourselves which was to figure out a way of improving the lab itself.
We noticed that some of the people were not that experienced with \texttt{C} and due to the performance heavy x264 application the small number of \rvex{} machines, testing during the lab was an absolute pain and took a lot of valuable time away.

The new assignments are placed in four categories:

\paragraph{getting your C going} \hfill \\
adder - write a C application that takes two integers as arguments and adds them, compile it for the VM

pow - write a C application that takes two integers a and b where $b \ge 0$ and calculates $a^b$ without the math library, compile and run it for the Microblaze FPGA

\paragraph{using file operations} \hfill \\
bin2c - create a binary to c-code converter, read in a binary file and print a character array with the bytes in it.

\paragraph{introducing rvex} \hfill \\
rvex-adder - write an rvex kernel that adds two integers

rvex-pow - write an rvex kernel that multiplies two integers, use it to calculate the power

\paragraph{abstracting rvex} \hfill \\
rvex-lib - write a simple interface for rvex.c and rvex.h containing the following functions
\begin{enumerate}
	\item rvexInitialize: initialize rvex object
	\item rvexDispose: clean up rvex object
	\item rvexBytecode: set the bytecode for the rvex
	\item rvexSeek: set the data cursor position
	\item rvexRead: read from data memory
	\item rvexWrite: write to data memory
	\item rvexGo: start the rvex and block until the operation is finished
\end{enumerate}

They should operate on an Rvex object that you must define\. hint: struct, extern.

Apply the rvex-lib to your rvex-adder and then to your rvex-pow application. Test if everthing still is working as intended.

\paragraph{real-life application} \hfill \\
rvex-x264 - modify x264 to use a kernel

You can supply a basic project setup with Makefiles. Each category can use its own introduction giving hints such as use open instead of fopen for memory mapped files.

The adder application is useful for its extreme simplicity.
It is easy to test and discover argument passing and endianess handling (for the rvex version).

The pow application is useful because it requires multiple calls to rvexGo.
This will remind students of resetting the data memory cursor for the rvex version.

\begin{lstlisting}[language=C,style=C,caption=pow - Microblaze part,label=lst:pow-microblaze]
#include <stdio.h>
#include <stdlib.h>
#include "rvex.h"
#include "bytecodes/mult.h"

int main(int argc, char **argv) {

	if (argc != 3) {
		printf("Use like this: `%s n c` to let the "
			"program calculate n^1..c\n", argv[0]);
		return -1;
	}

	rvexInit(&rvex0, bytecode, sizeof(bytecode),
		RVEX_0_INSTRUCTION_MEMORY_FILE,
		RVEX_0_DATA_MEMORY_FILE,
		RVEX_0_CORE_CTL_FILE,
		RVEX_0_CORE_STATUS_FILE);

	int n = atoi(argv[1]);
	int c = atoi(argv[2]);
	int temp = 1;
	while(c > 0) {
		rvexSeek(&rvex0, 0);
		rvexWrite(&rvex0, &n, sizeof(int));
		rvexWrite(&rvex0, &temp, sizeof(int));
		rvexGo(&rvex0);
		rvexRead(&rvex0, &temp, sizeof(int));
		c--;
	}

	printf("n^c = %d\n", temp);

	rvexDispose(&rvex0);

	return 0;
}
\end{lstlisting}

It is a simple program but uses all the functionality with rvex that you need for x264.

\begin{lstlisting}[language=C,style=C,caption=pow - kernel part,label=lst:pow-kernel]
int a = 0xAAA00A00, b = 0xBBB00B00, result = 0xCCC00C00;

#define FLIP_ENDI_32(a) ( \
		(((a)&0x000000FF) << 24) | \
		(((a)&0x0000FF00) << 8) | \
		(((a)&0x00FF0000) >> 8) | \
		(((a)&0xFF000000) >> 24) \
	)

int main () {
	a = FLIP_ENDI_32(a);
	b = FLIP_ENDI_32(b);
	result = a * b;
	result = FLIP_ENDI_32(result);
	return 0;
}
\end{lstlisting}

It is possible to distribute an already filled in rvex.h and leave the implementation to the students if they are running out of time.
At first it is not even necessary to give that much detail about a possible implementation.

\section{Conclusion}

We've identified a kernel function and changed the x264 program so the kernel
is executed on the \rvex. The kernel we used is the \satd function. Using our
abstracted \texttt{rvex*} functions you can write the instruction memory, write
and read the data memory and start the \rvex. Unfortunately we didn't see a
speedup, instead we even saw a decrease in performance. Assumably this is
because of the communication overhead. The \satd function is called many times.
Maybe by sending more data at once, and thus executing the \rvex less times
might improve the performance.

Using our \texttt{rvex*} functions, we were able to create different test
programs which helped us testing the communication and get to know the
platform, getting the endianness right and fixing multiple executions of the
\rvex.

All in all the lab was definitely fun. It felt really rewarding to hack
something together, inject some bytecode and see it working.

Regarding the results, we have some suggestions that we would have tried out
given enough time.

We still believe in the \rvex{} so we would try to reduce the performance loss
of the current bottleneck: communication. This can be done by reducing the
number of reads and writes to the \rvex{}. This means selecting a bigger part
of the application and send it over to the \rvex{} in its entirety.

We are interested in if the \rvex{} performance is dependent on the type of
program that is run. Is it more efficient for highly repeating code or does it
perform just as well for dynamic code with little repetition compared to the
normal CPUs?



%\bibliographystyle{plain}
%\bibliography{main}

\end{document}
