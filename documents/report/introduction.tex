\section{Introduction}

The assignment for this report was basically, find a kernel function from an
application and run it on the \rvex{}. To do this there are a few things we
need to know first:

\paragraph{Platform} The platform the program will run on is the ERA platform.
It consists out of the host processor, the MicroBlaze, accelerated with a
VLIW co-processor, the \rvex{}. For the co-processor computationally intensive
kernels can be extracted to achieve a performance increase.

\paragraph{Kernel function} The extracted piece of code is called the kernel.
The kernel needs to be compiled for the \rvex{} so that we can inject the
result, the bytecode, in to the co-processor. The rest of the code runs as
usual on a regular processor.

\paragraph{x264} The x264 program is the application we will try to improve
using the \rvex{}. x264 is a software library for encoding video streams
in the H.264 compression format.

The goal for this report is to extract the correct kernel, using profiling and
compile this for the \rvex{}. To execute the kernel on the \rvex{} there should
be communication between the processor and the co-processor.
