\section{Data Layout}

The \satd{} function has the following signature:

\begin{verbatim}
int x264_pixel_satd_8x4(
    pixel *pix1, int i_pix1,
    pixel *pix2, int i_pix2);
\end{verbatim}

This is a function that takes two arrays of pixels, something that has to do
with the indices of the pixels in the array, and returns a single integer.

From the implementation we can see that the data used by the array is always $4
\times 8 = 32$ pixels per array. This means that the data layout should have at
least two times 32 times the size of one pixel. The size of a pixel is defined
as \texttt{uint\_8}. If we choose \texttt{i\_pix1} and \texttt{i\_pix2} to
always be 8, we don't have to send those values, which saves sending data.
Setting those parameters to a fixed value has also the advantage that the array
indexes looked up by the function are always predictably 0 to 32

Also for the return value we should reserve 4 bytes, which is the size of an
integer. When we have calculated the result, we could write this back to the
first position, but we write the result after the data of the two pixel arrays.
The advantage for doing this is because it's slightly easier and clearer for future maintainers. 
